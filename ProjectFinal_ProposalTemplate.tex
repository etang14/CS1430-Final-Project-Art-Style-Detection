%%%%%%%%%%%%%%%%%%%%%%%%%%%%%%%%%%%%%%%%%%%%%%%%%%%%%%%%%%%%%%%%%%%%%%%%%%%%%%%%%%%%%%%%%%%%%%%%
%
% CSCI 1430 Project Proposal Template
%
% This is a LaTeX document. LaTeX is a markup language for producing documents.
% Your task is to answer the questions by filling out this document, then to 
% compile this into a PDF document. 
% You will then upload this PDF to `Gradescope' - the grading system that we will use. 
% Instructions for upload will follow soon.
%
% 
% TO COMPILE:
% > pdflatex thisfile.tex
%
% If you do not have LaTeX and need a LaTeX distribution:
% - Departmental machines have one installed.
% - Personal laptops (all common OS): http://www.latex-project.org/get/
%
% If you need help with LaTeX, come to office hours. Or, there is plenty of help online:
% https://en.wikibooks.org/wiki/LaTeX
%
% Good luck!
% James and the 1430 staff
%
%%%%%%%%%%%%%%%%%%%%%%%%%%%%%%%%%%%%%%%%%%%%%%%%%%%%%%%%%%%%%%%%%%%%%%%%%%%%%%%%%%%%%%%%%%%%%%%%
%
% How to include two graphics on the same line:
% 
% \includegraphics[width=0.49\linewidth]{yourgraphic1.png}
% \includegraphics[width=0.49\linewidth]{yourgraphic2.png}
%
% How to include equations:
%
% \begin{equation}
% y = mx+c
% \end{equation}
% 
%%%%%%%%%%%%%%%%%%%%%%%%%%%%%%%%%%%%%%%%%%%%%%%%%%%%%%%%%%%%%%%%%%%%%%%%%%%%%%%%%%%%%%%%%%%%%%%%

\documentclass[11pt]{article}

\usepackage[english]{babel}
\usepackage[utf8]{inputenc}
\usepackage[colorlinks = true,
            linkcolor = blue,
            urlcolor  = blue]{hyperref}
\usepackage[a4paper,margin=1.5in]{geometry}
\usepackage{stackengine,graphicx}
\usepackage{fancyhdr}
\setlength{\headheight}{15pt}
\usepackage{microtype}
\usepackage{times}
\usepackage{booktabs}

% From https://ctan.org/pkg/matlab-prettifier
\usepackage[numbered,framed]{matlab-prettifier}

\frenchspacing
\setlength{\parindent}{0cm} % Default is 15pt.
\setlength{\parskip}{0.3cm plus1mm minus1mm}

\pagestyle{fancy}
\fancyhf{}
\lhead{Final Project Proposal}
\rhead{CSCI 1430}
\rfoot{\thepage}

\date{}

\title{\vspace{-1cm}Final Project Proposal}

\begin{document}
\maketitle
\vspace{-1cm}
\thispagestyle{fancy}

\emph{Please make this document anonymous. Your team name should be anonymous.}

\textbf{Team name: \emph{HERE PLEASE}}

\emph{Note:} when submitting this document to Gradescope, make sure to add all other team members to the submission. This can be done on the submission page after uploading (top right).

If you need to find team members, please use the thread under `Final Project - Find Teammates' on Ed---pitch an idea!

\section*{Proposal Instructions}

For your project proposal, please submit a one-to-two page document answering the questions below.

\begin{itemize}
  \item What are the skills of the team members? Conduct a skill assessment!
  \item What is your project idea?  
  \item What is the socio-historical context that this project lives in? 
  \item Please list three groups of people that your project could impact, and describe how it could impact them. 
  \item What data will you use?
  \item What software/hardware will you use?
  \item Who will do what? [For anonymity, please use `'`Team member 1 will...'' or, alternatively, take on daring pseudonames.]
  \item How will you know whether you have made progress? What will you measure?
  \item What technical problems do you foresee or have?
  \item Is there anything that we can do to help? E.G., resources, equipment.
\end{itemize}

Feel free to use these as paragraph headings, and also please include any media, references, etc.

\section*{After Proposal Submission}

\subsection*{Proposal Swap}

After handing in your project proposal, your team will receive another team's proposal, and they will receive yours (remember to make your document anonymous).

Given the other team's proposal, your team must critique their understanding of their project's impact, and devise a list of potential socially-responsible computing concerns with their project. These should be written up in a document, and submitted on Gradescope the same day Progress Report 1 is due.

The other team will do the same for your team's proposal and project idea.

Your team will then receive the other team's critique of your project proposal. 
Your team must respond to this critique as a graded part of your Final Project Report.

\subsection*{TA Assignment}

After handing in your project proposal, your team will also be assigned a TA to assist you. 
You should aim to meet with your TA once a week; this replaces TA office hours.

If you haven't heard from your TA a few days after the project proposal handin, please make a private Ed post and let us know which team you're on.

In your first meeting with your TA, your goal is to have your idea sanity-checked:

\begin{itemize}
  \item Do you actually have the data?
  \item Do you actually have the compute?
  \item Is there code you need but don't have access to?
  \item Is there an area where you need help?
\end{itemize}

Some of these things will be outlined in your proposals, but talking through it with your TA as soon as possible will help you find potential road blocks and get the ball rolling.

\end{document}